% Table: Small-Sample Bias and Variance (Horizon 0)
\begin{table}[htbp]
\centering
\caption{Comparison to simulation \cite[Table 1]{DLP12}}
\label{tab:driessen_comparison}
\resizebox{\linewidth}{!}{%
\begin{tabular}{l ccccc cccccc}
\toprule
& \multicolumn{5}{c}{Market Beta ($\beta$)} & \multicolumn{6}{c}{Alpha ($\alpha$, monthly \%)} \\
\cmidrule(lr){2-6} \cmidrule(lr){7-12}
Scenario & True & Mean & SD & 25\% & 75\% & True & Mean & SD & 25\% & 75\% & \% Bound \\
\midrule
(S1) $n=20$, $\sigma=20\%$, Normal & 1.00 & 0.86 & 0.38 & 0.62 & 1.08 & 0.00 & 0.21 & 0.24 & 0.05 & 0.31 & 2\% \\
(S2) $n=50$, $\sigma=20\%$, Normal & 1.00 & 0.90 & 0.28 & 0.75 & 1.05 & 0.00 & 0.12 & 0.15 & 0.03 & 0.18 & 0\% \\
(S3) $n=50$, $\sigma=30\%$, Normal & 1.00 & 0.80 & 0.66 & 0.34 & 1.20 & 0.00 & 0.49 & 0.43 & 0.13 & 1.00 & 28\% \\
(S4) $n=50$, $\sigma=30\%$, Shifted LN & 1.00 & 0.85 & 0.66 & 0.35 & 1.26 & 0.00 & 0.42 & 0.42 & 0.08 & 0.90 & 22\% \\
(S5) $n=50$, $\sigma=30\%$, Shifted LN, Sim.~MKT & 1.00 & 0.87 & 0.69 & 0.41 & 1.24 & 0.00 & 0.48 & 0.50 & 0.06 & 1.00 & 32\% \\
(S6) $n=50$, $\sigma=30\%$, Shifted LN, Sim.~MKT, $\bar{\alpha}=0.5\%$ & 1.00 & 1.10 & 0.66 & 0.67 & 1.39 & 0.00 & 0.27 & 0.31 & 0.06 & 0.50 & 51\% \\
(S7) $n=50$, $\sigma=20\%$, Shifted LN, Sim.~MKT & 1.00 & 0.96 & 0.31 & 0.83 & 1.11 & 0.00 & 0.10 & 0.23 & -0.03 & 0.18 & 1\% \\
(S8) $n=50$, $\sigma=30\%$, Shifted LN, Sim.~MKT, Det.~Timing & 1.00 & 0.93 & 0.72 & 0.44 & 1.29 & 0.00 & 0.42 & 0.51 & 0.01 & 1.00 & 29\% \\
(S9) $n=50$, $\sigma=30\%$, Shifted LN, Sim.~MKT, $\bar{\alpha}=0.75\%$ & 1.00 & 0.97 & 0.67 & 0.53 & 1.30 & 0.00 & 0.38 & 0.41 & 0.06 & 0.75 & 41\% \\
\addlinespace
\multicolumn{12}{l}{\textit{Driessen et al. 2012, Table 1}} \\
Benchmark Estimates ($n=50$, $\sigma_{\text{monthly}}=30\%$) & 1.00 & 0.98 & 0.32 & 0.80 & 1.15 & 0.00 & 0.05 & 0.2 & -0.06 & 0.15 & n/a \\
\bottomrule
\end{tabular}%
}
\begin{minipage}{\linewidth}
\vspace{0.2cm}
\footnotesize \textit{Notes:} 
This table reports the small-sample properties of the generalized estimator for Horizon~0 (i.e., treating funds as held to maturity), mimicking the structure of Table~1 in \cite{DLP12}. We simulate 1,000 Monte Carlo draws. All scenarios use 20 vintage years (1986--2005) with 15 deals per fund; unless stated otherwise, $n=20$ funds per vintage, $\sigma=0.2$ monthly idiosyncratic volatility, and a normal error distribution. The simulation dimensions vary as follows: $n$ = number of funds per vintage-year portfolio; $\sigma$ = monthly idiosyncratic standard deviation; \textit{Normal} = Gaussian error term (default); \textit{Shifted LN} = shifted lognormal error term (cf.\ \cite{DLP12}); \textit{Sim.~MKT} = total market return drawn from a shifted lognormal (S\&P~500 1980--2003 moments) with a constant risk-free rate of 4\% annualized, instead of conditioning on the historical path; \textit{Det.~Timing} = deterministic (evenly spaced) deal investment dates instead of random; $\bar{\alpha}$ = upper bound on the alpha parameter in the optimization (default $\bar{\alpha} = 1\%$/month). The \textit{\%~Bound} column reports the fraction of draws where the alpha estimate reached the upper optimization bound~$\bar{\alpha}$. The final row presents the Benchmark Estimates obtained directly from Table~1 in \cite{DLP12} for comparative purposes. Alpha estimates are monthly and presented in percentages.
\end{minipage}
\end{table}
